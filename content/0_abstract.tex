\thispagestyle{empty}
\section*{Kurzdarstellung}
\label{sec:kurzdarstellung}
%Kurze Zusammenfassung der Arbeit, höchstens halbe Seite.
Open Source Projekte entwickeln Software für den öffentlichen Gebrauch. Diese soll auf unterschiedlichen Architekturen laufen können. Das Problem ist, dass Z Systeme - Mainframes von IBM - teuer sind und es ist schwierig Software für grundlegende Hardware ohne Zugriff darauf zu testen. Deshalb muss die Architektur S390x für Z Systeme für ausgewählte Open Source Projekte emuliert (nachgeahmt) und die entschschrechende CI/CD-Pipeline für automatisierte Tests integriert werden. Das wird mit Docker-Images realisiert, die in den Emulator QEMU integriert werden. Kubernetes wird als Beispiel-Projekt für Kubernetes-basierte Projekte als Grundlage aller containerisierten Kubernetes-Projekte verwendet. Ein weiteres Open Source Projekt ist Apache Cassandra, um Tests auf der Anwendungsschicht im Kubernetes-Stack zu repräsentieren. \\
Die minimalen Systemanforderungen müssen aus Hardware-Kostengründen für die Einrichtung innerhalb der CI/CD-Infrastruktur von Kubernetes und Apache Cassandra auch analysiert werden. Als letzter Schritt wird die automatische Emulation beider Projekte in die Test-Infrastruktur integriert, so dass diese Projekte für die Architektuk von Z Systemen aktiviert ist. Allgemein soll diese Methode für weitere Open Source Projekte in der Zukunft wiederverwendet werden können.



\section*{Abstract}
\label{sec:abstract}

Open-source-projects are developing software for public usage. That should be able to run on different architectures. The problem is that Z systems - mainframes by IBM - are expensive, and it is difficult to test software for essential hardware without access. Therefore, the architecture S390x for Z systems has to be emulated for chosen open-source-projects and integrated into their CI/CD pipeline for automated tests. That will be realized with Docker images integrated into the emulator QEMU. Kubernetes is used as an example project for Kubernetes-based projects as a foundation for all containerized Kubernetes projects. Another open-source project is Apache Cassandra to represent tests on the application layer in the
Kubernetes stack. \\
Minimal system requirements have to be analyzed for the setup inside of the CI/CD infrastructure of Kubernetes and Apache Cassandra because of hardware costs, too. As the last step, the automated emulation of both projects will be integrated into the test infrastructure, that these projects are enabled for the architecture of Z systems. Overall, this method should be
reapplied for further open-source-projects in the future, too.


