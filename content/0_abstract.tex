\thispagestyle{empty}
\section*{Kurzdarstellung}
\label{sec:kurzdarstellung}
%Kurze Zusammenfassung der Arbeit, höchstens halbe Seite.
Kubernetes ist eine Container-Plattform zur Orchestrierung mit unterschiedlichen Container-Runtimes, wie Docker, CRI-O und Podman für Hochverfügbarkeits-Cluster. Es gibt Hardware-Abhängigkeiten für die Software. Deshalb sollen alle Dienste und Container-Applikationen darauf basierend auch auf IBM Z Systemen laufen können. Die meisten Kubernetes-basierten Projekte sind Open-Source-Projekte. Sie haben ihre eigene Infrastruktur für automatisierte Tests mit Continuous Integration. IBM Z Hardware kann mit QEMU\/libvirt als Hypervisor emuliert werden. So kann der Quellcode immer getestetet werden, so wie es auch schon für die Architektur x86 durchgeführt wird. Außerdem werden auch Fehler schneller erkannt. Das erleichert es den einzelnen Communities Software für spezielle Hardware zu entwickeln. Die Emulation soll in sämtliche CI\/CD-Umgebungen der unterschiedlichen Open-Source-Projekte integriert werden können. Mainframes, wie IBM Z, verwenden die s390x-Architektur. Normalerweise haben Open-Source Communities keinen Zugriff für Tests zu solchen Systemen. Deshalb ist Emulation ein durchführbarer Weg diese Communities mit der Befähigung alternativer Hardware zu unterstützen.

\blindtext


\section*{Abstract}
\label{sec:abstract}
\emph{Only if thesis is written in English.}
Kubernetes exists as a container orchestration platform with different container runtimes as Docker, CRI-O and Podman for clustering. There is some hardware dependency for the software. Therefore, all services and container applications should be able to run based on that on IBM Z systems, too. Most Kubernetes based projects are open-source-projects. They have their own infrastructure for automated test environments (Continuous Integration). IBM Z hardware can be emulated with QEMU\/libvirt as a hypervisor. In this way, the source code can be tested continuously as it is for x86 architecture and can discover earlier possible bugs. That makes it easier for communities to develop software for special hardware. Emulation should be able to be integrated in all CI\/CD environments by different open-source projects. Mainframes as IBM Z are using the s390x architecture. Usually open-source communities don't have access to such systems for builds and tests. For this reason, emulation is a viable way to enable those communities supporting alternative architecture.
The first step is to understand their CI\/CD infrastructure and how to integrate emulation in their
configuration. The investigation on various open-source-projects can help to identify a common pattern how to integrate emulation. Hardware emulation requires a lot of performance. Therefore, minimal system requirements have to be analyzed for the emulation in the next step. This hardware emulation will added into the test environment of both open-source projects Kubernetes and Apache Cassandra then. The goal of this Bachelor Thesis is to apply and integrate emulation for s390x architecture into the infrastructure for various open-source projects. That can be reapplied for other open-source projects then, too. 


\blindtext
