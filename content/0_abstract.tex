\thispagestyle{empty}
\section*{Kurzdarstellung}
\label{sec:kurzdarstellung}
%Kurze Zusammenfassung der Arbeit, höchstens halbe Seite.
Open-Source-Projekte entwickeln Software, die frei verfügbar ist. Diese Communitys lassen automatisierte Tests gegen jede Code-Änderung laufen, um die bester Software-Qualität zu garantieren. 
Diese Tests sollen auf unterschiedlichen Architekturen laufen können. Es ist schwierig, Software für grundlegende Hardware ohne Zugriff darauf - wegen des hohen Preises, zu testen. Deshalb muss die Architektur S390x für Z Systeme für ausgewählte Open-Source-Projekte emuliert (nachgeahmt) und in die entsprechende CI/CD-Pipeline eingefügt werden. Das sollte mit modernen und schnellen Deployment-Methoden realisiert werden, die in den Emulator QEMU integriert werden. Kubernetes wird als Beispiel-Projekt für containerisierte Projekte mit der Sicht, es als Grundlage für alle darauf aufbauenden Projekte verwendet. Ein weiteres Open-Source-Projekt ist Apache Cassandra, um Tests auf der Anwendungsschicht im Kubernetes-Stack zu repräsentieren. \\
Die minimalen Systemanforderungen müssen für die Einrichtung innerhalb der CI/CD-Infrastruktur beider Projekte mit einem Blick auf die Minimierung an Festplattenplatz, Memory und CPU-Verbrauch auch analysiert werden. Als letzter Schritt wird die automatische Emulation beider Projekte in die Test-Infrastruktur integriert, so dass diese Projekte für die Architektur von Z Systemen aktiviert ist. Allgemein soll diese Methode für weitere Open Source Projekte in der Zukunft wiederverwendet werden können.



\section*{Abstract}
\label{sec:abstract}

Open-source-projects are developing software which is free available. These communities are running automated tests against every code change in order to guarantee the best software quality. 
These tests should be able to run on different architectures. It is difficult to test software for essential hardware without access because of the high price. Therefore, the architecture \gls{s390x} for \gls{Z systems} has to be emulated for chosen open-source-projects and included into their \gls{CI/CD} pipeline. That should be realized with modern and fast deployment methods integrated into the \gls{emulator} QEMU. Kubernetes is used as an example project for \gls{containerized} projects in the point of view as the foundation for all establishing projects. Another open-source project is Apache Cassandra to represent tests on the \gls{application layer} in the Kubernetes stack. \\
Minimal system requirements have to be analyzed for the setup inside of the CI/CD infrastructure of both projects with an eye on the minimization of space, memory and cpu usage for deployments, too. As the last step, the automated emulation of both projects will be integrated into the test infrastructure, that these projects are enabled for the architecture of Z systems. Overall, this method should be reapplied for further open-source-projects in the future, too.


