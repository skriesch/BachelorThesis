\chapter{Prequisites}\label{ch:prequisites}


Different software is necessary to run qemu or docker for multiple architectures. Therefore, docker and qemu should be installed. Additionally, qemu-user-static \footnote{\url{https://github.com/multiarch/qemu-user-static}} and binfmt\_misc \footnote{\url{https://www.kernel.org/doc/html/latest/admin-guide/binfmt-misc.html}} are important for running multi-architecture containers. \\
\textbf{root permissions} are necessary for installation, configuration and running emulated systems.

It is possible to install packages as binfmt-support and qemu-user-static by different Linux distributions, but it is recommended to use the latest possible version for s390x. \\

\section{Registration of qemu-s390x}

The kernel module binfmt\_misc can be mounted with the following command: \\
\lstinline!mount binfmt_misc -t binfmt_misc /proc/sys/fs/binfmt_misc!

Latest stable releases of qemu-user-static can be found under \url{https://github.com/multiarch/qemu-user-static/releases/}. The release v5.0.0-2 is used for the project and downloaded with \lstinline!wget https://github.com/multiarch/qemu-user-static/releases/download/v5.0.0-2/x86_64_qemu-s390x-static.tar.gz! for the special version of qemu-s390x-static on x86. That is extracted to the directory \path{/usr/bin/} with the command \lstinline!sudo tar -xvzf x86_64_qemu-s390x-static.tar.gz -C /usr/bin/! then. \\

s390x binaries have to be registered for s390x. That is done with the following commands: \lstinline!sudo -i! and \begin{lstlisting}[language=bash, numberstyle=\myHexNumber, caption={Register s390x binaries},captionpos=b]
# echo ':qemu-s390x:M::\x7fELF\x02\x02\x01\x00\x00\x00\x00\x00\x00\x00\x00\x00\ 
x00\x02\x00\x16:\xff\xff\xff\xff\xff\xff\xff\x00\xff\xff\xff\xff\xff\xff\xff\xff\
xff\xfe\xff\xff:/usr/bin/qemu-s390x-static:OCF' > /proc/sys/fs/binfmt_misc/register
\end{lstlisting}

Docker is configured to build only for the own architecture, which is x86 at open-source projects. The "Experimental" flag exists for new available features which are not ready for production. So you can build docker images for s390x on x86 then. That can be added with the following:

\begin{lstlisting}[language=bash,caption={Docker Experimental Flag},captionpos=b]
{
"experimental": enabled
} >> /etc/docker/daemon.json
\end{lstlisting}

After a restart of the docker daemon there should be listed this flag in the command  \lstinline!docker version!. An alternative way is to export this flag as an environment variable in the shell with  \lstinline!export DOCKER_CLI_EXPERIMENTAL=enabled! to enable it. Now it is possible to build docker images for s390x on x86 with  \lstinline!docker build --platform=linux/s390x -t image-example:s390x .! based on a Dockerfile in the existing directory.

 

 

\section{Cross-Compilation of a s390x Linux Kernel on x86}

QEMU needs a built Linux kernel to start a system. These packages are necessary: \\
\textbf{bison, flex, gcc, gcc-s390x-linux-gnu, libssl-dev} \\
One kernel has to be downloaded from kernel.org with the following command then:  \lstinline!wget https://git.kernel.org/torvalds/t/linux-5.7-rc7.tar.gz!\\
It can be extracted with \lstinline!tar -xf  linux-5.7-rc7.tar.gz!. The Makefile is in the directory linux-5.7-rc7. Therefore the command \lstinline!make ARCH=s390 defconfig localyesconfig! has to be executed to create the default configuration for s390x and  \lstinline!make ARCH=s390 CROSS_COMPILE=s390x-linux-gnu- -j6! for the compilation. The kernel has the name bzImage in the directory \path{arch/s390/boot/} then and can be used in the qemu command.

\section{Optimized QEMU Command}

Every additional device requires additional performance and time for starting the system. 
So the systems requirements had to be figured out that system requirements are minimal for every open-source project and for running tests on it. 
That counts for the number of CPUs, too. \\

The kernel option is receiving the path to the built s390x kernel. 
The option -m is available to add the minimal guest memory matching the system requirements of every open-source project. 
\textbf{-nodefaults} is deactivating default additional devices activated in QEMU. 
Only the console is necessary for receiving an output and debugging. 
So that is added as a device. Cassandra as a project does not need any network interface or parallelism. The 
option \textbf{-nographic} is responsible for not adding any graphical interface. 
So we save system requirements. The option \textbf{-smp} is the minimal number of CPUs for the guest. 
The file system of containers can be loaded as a hard disk with the option \textbf{-hda} which is explained in every chapter of a special open-source project. 
That is the ideal option to mount a minimal file system for every application or system. 
\path{/dev/vda} is the partition name and rdinit is used for using Bash as a default shell. \\


\begin{lstlisting}[language=bash,caption={Optimized QEMU Command},captionpos=b]
/usr/bin/qemu-system-s390x -kernel bzImage -m 4G -M s390-ccw-virtio -nodefaults -device sclpconsole,chardev=console -parallel none -net none -chardev stdio,id=console,signal=off,mux=on -mon chardev=console -nographic -smp 3 -hda /data/kub-container.img --append 'root=/dev/vda rw console=ttyS0 rdinit=/bin/bash'
\end{lstlisting}

