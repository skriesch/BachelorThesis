\chapter{Kubernetes}\label{ch:kubernetes}

\section{Overview}

Kubernetes is a container platform for high availability clusters.
There exist plugins for integration tests for Kubernetes with the name kubetest\footnote{\url{https://kubetest.readthedocs.io/en/latest/}}. They contain conformance tests, as well as e2e tests (end-to-end), too.
That can be all built and executed on the system. Therefore, a Dockerfile for setting up Kubernetes and building tests with Go is necessary. The challenge is, that 2 big Github repositories have to be cloned and integrated into the docker image. That is using a lot of space. The solution is using a multi staging Dockerfile. 
So 2 different Dockerfiles are used in one Dockerfile and one of them is used for building. The other one is used for the installation and testing with built tests. At the end the size of the docker image has got only the size of the test image regardless of the repository size in the mother Dockerfile.

\section{Multi Staging Dockerfile}

A multi-staging Dockerfile is using different systems in one Dockerfile for different stages. These systems are receiving special names as indicators with "AS" behind the "FROM" with the base image name. 
Default this feature is used for building applications in one stage and executing the copied application in another stage. The same counts for cloning Github repositories and building binary files based on it. On this way, a lot of space is saved.
Concluding, the docker image has got only the size of the executing system with the application file (without all the code). 
That is an "experimental feature"  at the moment. Therefore the \textbf{experimental flag} is necessary to export or set before using it (see \ref{Multi-Architecture-Images} Multi-Architecture Images). 

\section{Installation}

Kubernetes needs a lot of packages for running and for tests. That will be all installed with the RUN command.
\url{apt.kubernetes.io} has got later versions of Kubernetes than the Ubuntu repository. Therefore this repository has to be added to Ubuntu. kub-build is the name of the mother Dockerfile to be able to copy needed files and directories from there. \\
In the intallation part of the Dockerfile the Kubernetes repository \url{https://apt.kubernetes.io} for Debian packages has to be registered together with with the gpg key used by  \url{https://packages.cloud.google.com/apt/doc/apt-key.gpg}.
A Dockerfile is installing only necessary packages. Therefore, apt-transport-https, apt-utils, curl, git, ca-certificates, gnupg-agent and software-properties-common have to be installed first for the following installation.
The system requires the lates update with \lstinline!apt-get update! after the registration of the additional repository besides of the default Ubuntu repositories imported with the base Ubuntu image "s390x/ubuntu:18.04" in the FROM command. After that the packages docker.io, kubelet and kubeadm can be installed from kubernetes.io. \textbf{Docker.io} contains the container engine docker with all docker commands. CRI/O or containerd would be allowed, too. 
The Docker daemon will be used because that is the main used container engine of the Kubernetes project and all tests are running withit. \\ \textbf{Kubelet} is the primary node agent running on each node. He is responsible that different containers can run together in a pod. Pods are deployable units defined in JSON or a yaml file. 
They include one or a group of containers with shared storage and network resources. Hosting of distributed systems  with different services in different containers can work together. Consequential one pod is soething as one "logical host". \\
\textbf{Kubeadm} is the administration tool to setup clusters. It is necessary to upgrade Kubernetes to other versions, too. Clusters can be initialized. The network can be configured and the command \textbf{kubectl} (Kubernetes Control Plane) for adding additional nodes to a cluster can be initialized. \\
\lstinline!apt-mark hold! is keeping these special versions of kubelet, kubeadm and kubectl. 

\begin{figure}[H]
\centering
\begin{boxedverbatim}
FROM s390x/ubuntu:18.04 AS kub-build
 
# The author
MAINTAINER Sarah Julia Kriesch <sarah.kriesch@ibm.com>

#Installation
RUN echo "Installing necessary packages" && \
apt-get update && apt-get install -y \
apt-transport-https \
apt-utils \
systemd \
curl \
git \
ca-certificates \
gnupg-agent \
software-properties-common \
&& curl -s https://packages.cloud.google.com/apt/doc/apt-key.gpg | apt-key add - \
&& echo "deb https://apt.kubernetes.io/ kubernetes-xenial main" \
> /etc/apt/sources.list.d/kubernetes.list \
&& apt-get update && apt-get install -y \
docker.io \
kubelet \
kubeadm \
&& apt-mark hold kubelet kubeadm kubectl \
&& apt-get clean \
&& rm -rf /var/lib/apt/lists/* /tmp/* /var/tmp/* \
&& systemctl enable docker 
\end{boxedverbatim}
 \caption{Kubernetes Installation}
    \label{kubernetes-installation}
\end{figure}

\section{Installation of the Ultimate Go}

There were some issues with older Go versions as 1.10 during building tests for Kubernetes. Therefore a higher version (min. 1.13) should be used. It is recommended to use the ultimate go version for last versioned Kubernetes tests. It is possible to receive the version number of the ultimate go release with the command \\ 
\lstinline!curl https://golang.org/VERSION?m=text!. \\ 
This version number has to be included before linux-s390x.tar.gz for downloading the special s390x archive from the go directory by \url{dl.google.com}. Then the version number has to be called with curl inside of another curl command with the whole path to the special tar archive on \url{dl.google.com}. Every tar archive has got the same structure for every version (\lstinline!$version.$architecture.tar.gz!). On this way the latest version of Go is integrable into the curl command that it can be installed. Directories for bin, pkg and src have to be created after extracting this tar archive in the \path{/root/} directory. They are not integrated in the tar archive. \\

The environment variables for GOROOT, GOPATH and PATH have to be set with ENV on the top of the Dockerfile for successful builds later. PWD is added because Github repositories have to be cloned to this directory. \\

The ENV variables will be on the top of the Dockerfile. The part for the "Installation of Go" will be attached to the end of the Kubernetes installation part.

\begin{figure}[H]
\centering
\begin{boxedverbatim}
ENV GOROOT=/root/go
ENV GOPATH=/root/go
ENV PATH=$GOPATH/bin:$PATH
ENV PATH=$PATH:$GOROOT/bin
ENV PWD=/root/go/src/

#Installation of latest GO
&& echo "Installation of latest GO" && \
curl "https://dl.google.com/go/ \
$(curl https://golang.org/VERSION?m=text).linux-s390x.tar.gz" \
| tar -C /root/ -xz \
&& mkdir -p /root/go/{bin,pkg,src} 
\end{boxedverbatim}
 \caption{Go Installation}
    \label{go-installation}
\end{figure}

\section{Building Tests}
After a successful installation of go, it is possible to build and install the Kubernetes test environment.
At first the directory k8s.io has to be created because kubernetes-tests are looking for this directory as a mother directory. The repository test-infra by the Kubernetes project has to be cloned to there. 
The repository \textbf{test-infra} contains all tests for Kubernetes provided by the community. 
These can be used of course and are updated continuously. That is the reason to clone this repository inside of the Dockerfile. Inside of this test-infra directory kubetest can be installed with \textbf{go install}. 
That is downloading all available Kubernetes-Tests. So you can use them to test the own Kubernetes cluster and the used software. \\

The name of the most relevant tests for the Kubernetes community is "conformance tests". These conformance tests are executed with e2e.test  which can be built with make inside of the kubernetes repository. 
Therefore this repository has to be cloned to k8s.io, too. These tests certify the software to comply regular standards. Only with complying these standards, Kubernetes software is allowed to become Kubernetes certified\footnote{https://github.com/cncf/k8s-conformance}. 


\begin{figure}[H]
\centering
\begin{boxedverbatim}
&& cd $PWD \
#Clone test-infra
&& mkdir -p $GOPATH/src/k8s.io \
&& cd $GOPATH/src/k8s.io \
&& git clone https://github.com/kubernetes/test-infra.git \
/root/go/src/k8s.io/test-infra \
&& cd /root/go/src/k8s.io/test-infra/ \
#Install kubetest
&& GO111MODULE=on go install ./kubetest \
#Build test binary
&& git clone https://github.com/kubernetes/kubernetes.git  \
/root/go/src/k8s.io/kubernetes \
&& cd /root/go/src/k8s.io/kubernetes/
 
CMD make WHAT="test/e2e/e2e.test vendor/github.com/onsi/ginkgo/ginkgo cmd/kubectl"
\end{boxedverbatim}
 \caption{Test Building for Kubernestes}
    \label{test-building}
\end{figure}

\subsection{E2e.test}

\section{Run in the Main Dockerfile}

\section{Integration into CI/CD}