\chapter{Cassandra}\label{ch:cassandra}

\section{Overview}

\section{Deployment}

IBM is offering a Dockerfile\footnote{\url{https://github.com/linux-on-ibm-z/dockerfile-examples/tree/master/ApacheCassandra}} for Apache Cassandra especially for Z systems with the latest version on Github. This file can be cloned to the system and the Docker image will be built with the command  \lstinline!docker build --platform=linux/s390x --squash -t cassandra:s390x .! in the directory with the Cassandra Dockerfile  for the architecture s390x. \textbf{squash} is an option to compress a Docker image and combine commands in a Dockerfile automatically. The prequisites for building s390x images on x86 are set duiring the emulation preparation. The command \lstinline!docker images! has to show the registered Dockerimage with the name cassandra:390x then. It should be possible to integrate this Docker image into the qemu command. Therefore, a qemu-image will be created with an rounded given size besides of the Docker image in the  \lstinline!docker images! command. So the command  \lstinline!qemu-img create -f raw cassandra.img 2G! can be used. This image needs any Linux file system because QEMU does not know the Docker file system. The image is formated with the command  \lstinline!mkfs.ext4 -F cassandra.img! then. For receiving the file system of the docker image a directory with the name rootfs has to be crated and the command
\lstinline!docker export $(docker create cassandra:s390x) | tar -C "rootfs" -xvf -! is exporting the docker image into the directory rootfs. Following transfers the content of rootfs into the image cassandra.img.

\begin{lstlisting}[language=bash,caption={Mount rootfs},captionpos=b]
mkdir /mnt/rootfs
mount -o loop cassandra.img /mnt/rootfs
cp -r rootfs/* /mnt/rootfs/.
 \end{lstlisting}

Now it is possible to run the system with Cassandra: \\
\begin{lstlisting}[language=bash,caption={Run Cassandra},captionpos=b]
 /usr/bin/qemu-system-s390x -kernel bzImage -m 40G -M s390-ccw-virtio -nodefaults -device sclpconsole,chardev=console -parallel none -net none -chardev stdio,id=console,signal=off,mux=on -mon chardev=console -nographic -smp 3 -hda /data/dockerfile-examples/ApacheCassandra/cassandra.img --append 'root=/dev/vda rw console=ttyS0 rdinit=/bin/bash' 
 \end{lstlisting}
 
\section{Workaround because of an Issue with JVMs during the Build with BuildX}

\section{Start of the application and tests}

A script has been written to start Cassandra and to run tests. Java and Cassandra have to be started here. The Icinga monitoring check has been used for experiments.
