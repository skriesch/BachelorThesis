\chapter{Summary}\label{ch:summary}

This Bachelor Thesis has introduced a new method to deploy applications and emulate the s390x architecture with a combination of the emulator \gls{QEMU} together with the container technology Docker.
If binfmt-support is enabled on any system, it is possible to use \gls{QEMU} with qemu-user-static for emulations of multiple system architectures. 
The challenge was that open-source communities do not support the architecture of IBM Z systems because of missing access for tests. The presented solution can be embedded in the \gls{CI/CD} infrastructure of open-source projects. 

A "Docker in Docker" solution has been available before the Bachelor Thesis project. "Kubernetes in Docker" is applied by the Kubernetes community for tests. 
Therefore, it is possible to upgrade Kubernetes setups inside of a container to a chosen version with a pre-installed Docker. A test environment has been created on this base with a Dockerfile for the installation and another Dockerfile. The he evaluation of \gls{CI/CD} systems found that Prow can not be utilized for emulations due to its containerized Kubernetes concept. 
Some research has yielded the result that Prow has been developed based on Jenkins and Jenkins X. Before the invention of Prow, the concept with Kubernetes based deployments had been  tried based on Jenkins with a Kubernetes plugin. Jenkins can deploy emulations and container systems. 
All facts considered, the emulated system with the Docker image can be tested with Jenkins and then transferred to Prow with the same configuration to.
Tests have been contained because of the extension of the study. A network interface and specific open ports have to be attached to the \gls{QEMU} emulation and analyzed. 

CircleCI works with container technologies and virtualization/ emulation technologies. 
Therefore, the emulated setup of Apache Cassandra could be integrated into the \gls{CI/CD} pipeline of the community. However, it fails at the moment because of a JVM issue in \gls{QEMU} for s390x during the build process. 
There exists a development workaround with a Docker build process on IBM Z and the integration into the local registry after the synchronization to an x86 system. Afterwards, the emulation with the integrated Docker image has performed as described.

The approach with a combination of \gls{QEMU} emulations and integrated Docker image can be applied for additional \gls{CI/CD} pipelines of open-source projects. Ultimately, this method provides an excellent environment for fast software deployments in general. 
Multiple system architectures can be emulated on different hardware. Following more system architectures can be supported by software vendors.