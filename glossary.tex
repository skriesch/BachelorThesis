\newglossaryentry{IBM Z systems}
{
    name={IBM Z systems},
    description={Mainframe hardware by IBM}
}

\newglossaryentry{s390x}
{
    name={s390x},
    description={Mainframe system architecture}
}

\newglossaryentry{z/OS}
{
    name={z/OS},
    description={One traditional mainframe operating system by IBM for Z systems}
}

\newglossaryentry{KVM}
{
    name={KVM},
    description={Kernel-based Virtual Machine is a open-source virtualization environment based on the Linux Kernel}
}

\newglossaryentry{XEN}
{
    name={XEN},
    description={Type-1 hypervisor for virtualization integrated into different commercial and open-source applications}
}

\newglossaryentry{QEMU}
{
    name={QEMU},
    description={QEMU (Quick EMUlator) is a free and open-source emulator and virtualizer to perform hardware virtualization}
}

\newglossaryentry{LPAR}
{
    name={LPAR},
    description={Mode with logical partition to divide a mainframe into one or more hardware resources for different running operating systems}
}

\newglossaryentry{Linux}
{
    name={Linux},
    description={Free operating system developed by different communities and available as different distributions}
}

\newglossaryentry{instruction}
{
    name={instruction},
    description={A single operation of a processor given by a program statement to be performed by the computer}
}

\newglossaryentry{database management system}
{
    name={database management system},
    description={System software for managing databases}
}

\newglossaryentry{Sparc32}
{
    name={Sparc32},
    description={32 bit architecture of Sparc (Scalable Processor Architecture) developed by Sun Microsystems}
}

\newglossaryentry{Sparc64}
{
    name={Sparc64},
    description={64 bit architecture of Sparc (Scalable Processor Architecture) developed by Sun Microsystems}
}

\newglossaryentry{MIPS}
{
    name={MIPS},
    description={MIPS (Microprocessor without Interlocked Pipelined Stages) is a load-store architecture developed by MIPS Technologies}
}

\newglossaryentry{PowerPC}
{
    name={PowerPC},
    description={Performance optimization with enhanced RISC Performance Chip is a RISC microprocessor architecture designed by IBM, Apple and Motorola}
}

\newglossaryentry{ARM}
{
    name={ARM},
    description={Architecture by Advanced RISC Machines (ARM) developed for System on Chips (SoC)}
}

\newglossaryentry{mainframe}
{
    name={mainframe},
    description={A large computer with extensive capabilities and resources which is able to execute millions of transactions in parallel}
}

\newglossaryentry{binary translation}
{
    name={binary translation},
    description={A method in which the machine language code of an application for the guest operating system is translated to the machine language code of the host operating system}
}

\newglossaryentry{x86}
{
    name={x86},
    description={Architecture of a default PC with processors by Intel or AMD}
}
\newglossaryentry{CI/CD}
{
    name={CI/CD},
    description={Continuous Integration/Continuous Delivery is a method for automated tests with integrated builds in the software development}
}

\newglossaryentry{emulator}
{
    name={emulator},
    description={Software for emulating different architectures of other systems for running software on it}
}

\newglossaryentry{containerized}
{
    name={containerized},
    description={Splitting programs into different services and deploying it connected in many containers as a single application}
}

\newglossaryentry{container}
{
    name={container},
    description={A a runtime instance consisting of a container image, an execution environment and a standard set of instructions}
}

\newglossaryentry{pod}
{
    name={pod},
    description={A group of one or multiple containers that are running on a Kubernetes cluster}
}

\newglossaryentry{registry}
{
    name={registry},
    description={A public or private container image storage and distribution service with built container images}
}

\newglossaryentry{NoSQL}
{
    name={NoSQL},
    description={A class of database management systems that consist of non-relational, distributed data stores}
}
\newglossaryentry{application layer}
{
    name={application layer},
    description={The layer in the software stack wich is responsible for running applications}
}

\newglossaryentry{Kubernetes stack}
{
    name={Kubernetes stack},
    description={Kubernetes split into different layers from the container environment until the application layer}
}

\newglossaryentry{library}
{
    name={library},
    description={A suite of reusable code inside of a programming language for software development}
}

\newglossaryentry{Go}
{
    name={Go},
    description={Programming language much used for container software}
}

\newglossaryentry{Java}
{
    name={Java},
    description={Programming language much used for Enterprise software development}
}

\newglossaryentry{monitoring system}
{
    name={monitoring system},
    description={A hardware/ software component to monitor system resources regarding availability, usage and performance}
}

\newglossaryentry{cluster}
{
    name={cluster},
    description={Resources as worker nodes, networks, and storage devices that keep apps highly available or performant}
}

\newglossaryentry{deployment}
{
    name={deployment},
    description={A system setup with pre-defined configuration properties and installed packages}
}

\newglossaryentry{JVM}
{
    name={JVM},
    description={Java Virtual Machine enables running Java applications compiled to Java bytecode}
}
\newglossaryentry{Linux kernel}
{
    name={Linux kernel},
    description={Software by the Linux community containing all important drivers for running the operating system}
}

\newglossaryentry{scaling}
{
    name={scaling},
    description={Distributing processes to different cores of a system and executing there}
}

\newglossaryentry{virtual machine}
{
    name={virtual machine},
    description={A virtual machine is a system running in a hypervisor as a separate system on another system}
}

\newglossaryentry{hypervisor}
{
    name={hypervisor},
    description={Software that deploys and run multiple guest virtual machines on real hardware}
}


\newglossaryentry{binary}
{
    name={binary},
    description={Executable application built in a single file}
}

\newglossaryentry{CPU}
{
    name={CPU},
    description={The Central Processing Unit is a main component of computers for executing instructions}
}

\newglossaryentry{memory}
{
    name={memory},
    description={The place inside of a computer for saving data before writing on the hard disk}
}

\newglossaryentry{RAM}
{
    name={RAM},
    description={Random-Access Memory in memory inside of a computer typically for storing working data and machine code}
}

\newglossaryentry{Dockerfile}
{
    name={Dockerfile},
    description={Base file with all installations and configurations for a container image}
}

\newglossaryentry{configuration management}
{
    name={configuration management},
    description={A process for maintaining computer systems, servers and software in a defined consistent state}
}

\newglossaryentry{Docker image}
{
    name={Docker image},
    description={A Docker file system with execution parameters that are used within a container runtime to create a Docker container}
}

\newglossaryentry{Docker daemon}
{
    name={Docker daemon},
    description={Manages and organizes objects inside of the container engine Docker}
}

\newglossaryentry{layer}
{
    name={layer},
    description={A container file system consist of layers, where the changed version is layered on top of the parent container image to create the new image}
}

\newglossaryentry{container engine}
{
    name={container engine},
    description={The foundation for managing and organizing containers}
}

\newglossaryentry{package}
{
    name={package},
    description={Software offered comprimized and with dependencies to other software by different Linux distributions}
}

\newglossaryentry{shell}
{
    name={shell},
    description={Terminal of a Linux/Unix system for entering commands}
}

\newglossaryentry{bash}
{
    name={bash},
    description={Unix shell and command language integrated in all Linux operating systems}
}

\newglossaryentry{bug}
{
    name={bug},
    description={A problem in software which results in an error}
}

\newglossaryentry{static}
{
    name={static},
    description={dependencies resolved during the compile time and copied into a executable target application}
}

\newglossaryentry{archive}
{
    name={archive},
    description={A compressed directory saved as a single file}
}

\newglossaryentry{issue}
{
    name={issue},
    description={A bug report because of a failure in the software}
}

\newglossaryentry{root permissions}
{
    name={root permissions},
    description={Administrator access permissions with all privileges for a computer system based on Linux/Unix}
}

\newglossaryentry{configuration}
{
    name={configuration},
    description={Saved settings for an application or computer program}
}

\newglossaryentry{mounted}
{
    name={mounted},
    description={Integrated into a system with the mount command}
}

\newglossaryentry{multi-architecture images}
{
    name={multi-architecture images},
    description={Images based on system builds usable for multiple system architectures}
}

\newglossaryentry{cross-compilation}
{
    name={cross-compilation},
    description={Method for compiling code for a different computer system or architecture}
}

\newglossaryentry{block device}
{
    name={block device},
    description={A peripheral device that handles data in blocks such as a disk}
}

\newglossaryentry{guest system}
{
    name={guest system},
    description={An additional virtual machine which is hosted on a local physical system}
}

\newglossaryentry{JSON}
{
    name={JSON},
    description={The JavaScript Object Notation is an open standard file format for saving data structured}
}

\newglossaryentry{yaml}
{
    name={yaml},
    description={A human-readable data-serialization language commonly used for configuration files or system deployment definitions}
}

\newglossaryentry{initrd}
{
    name={initrd},
    description={The initial RAM disk is the first file system (incl. executables) loaded in a Linux operating system when the root file system is available}
}

\newglossaryentry{Ubuntu}
{
    name={Ubuntu},
    description={Linux distribution maintained by Canonical}
}

\newglossaryentry{hard disk}
{
    name={hard disk},
    description={The whole operating system with data is written and saved on this component of the computer system}
}

\newglossaryentry{repository}
{
    name={repository},
    description={A persistent storage area for data or other application resources, and used for software development}
}

\newglossaryentry{Github}
{
    name={Github},
    description={Software version control system by Microsoft for free usage}
}

\glsaddall
