\newglossaryentry{IBM Z systems}
{
    name={IBM Z systems},
    description={Mainframe hardware by IBM}
}

\newglossaryentry{s390x}
{
    name={s390x},
    description={Mainframe system architecture}
}

\newglossaryentry{z/OS}
{
    name={z/OS},
    description={One traditional mainframe operating system by IBM for Z systems}
}

\newglossaryentry{KVM}
{
    name={KVM},
    description={Kernel-based Virtual Machine is a open-source virtualization environment based on the Linux Kernel}
}

\newglossaryentry{XEN}
{
    name={XEN},
    description={Type-1 hypervisor for virtualization integrated into different commercial and open-source applications}
}

\newglossaryentry{QEMU}
{
    name={QEMU},
    description={QEMU (Quick EMUlator) is a free and open-source emulator and virtualizer to perform hardware virtualization}
}

\newglossaryentry{LPAR}
{
    name={LPAR},
    description={Mode with logical partition to divide a mainframe into one or more hardware resources for different running operating systems}
}

\newglossaryentry{Linux}
{
    name={Linux},
    description={Free operating system developed by different communities and available as different distributions}
}

\newglossaryentry{x86}
{
    name={x86},
    description={Architecture of a default PC}
}


\newglossaryentry{mainframe}
{
    name={mainframe},
    description={Large computer which is able to execute millions of transactions in parallel}
}


\newglossaryentry{CI/CD}
{
    name={CI/CD},
    description={Continuous Integration/Continuous Delivery is a method for automated tests in the software development}
}

\newglossaryentry{emulator}
{
    name={emulator},
    description={Software for emulating different architectures of other systems for running software on it}
}

\newglossaryentry{containerized}
{
    name={containerized},
    description={Splitting programs into different services and deploying it connected in many containers as a single application}
}

\newglossaryentry{container}
{
    name={container},
    description={A standard unit with the minimum of mandatory software}
}

\newglossaryentry{pod}
{
    name={pod},
    description={An association of multiple containers with services for one application}
}

\newglossaryentry{registry}
{
    name={registry},
    description={Built container images are registered and maintained there for general usage}
}

\newglossaryentry{application layer}
{
    name={application layer},
    description={The layer in the software stack wich is responsible for running applications}
}

\newglossaryentry{Kubernetes stack}
{
    name={Kubernetes stack},
    description={Kubernetes splitted into different layers from the container environment until the application layer}
}

\newglossaryentry{library}
{
    name={library},
    description={A suite of reusable code inside of a programming language for software development}
}

\newglossaryentry{Go}
{
    name={Go},
    description={Programming language much used for container software}
}

\newglossaryentry{Java}
{
    name={Java},
    description={Programming language much used for Enterprise software development}
}

\newglossaryentry{JVM}
{
    name={JVM},
    description={Java Virtual Machine enables running Java applications compiled to Java bytecode}
}
\newglossaryentry{Linux kernel}
{
    name={Linux kernel},
    description={Software by the Linux community containing all important drivers for running the operating system}
}

\newglossaryentry{scaling}
{
    name={scaling},
    description={Distributing processes to different cores of a system and executing there}
}

\newglossaryentry{virtual machine}
{
    name={virtual machine},
    description={A virtual machine is a system running in a hypervisor as a separate system on another system}
}

\newglossaryentry{hypervisor}
{
    name={hypervisor},
    description={Software that deploys and run multiple guest virtual machines on real hardware}
}


\newglossaryentry{binary}
{
    name={binary},
    description={Executable application built in a single file}
}

\newglossaryentry{CPU}
{
    name={CPU},
    description={The Central Processing Unit is a main component of computers for executing instructions}
}

\newglossaryentry{memory}
{
    name={memory},
    description={The place inside of a computer for saving data before writing on the hard disk}
}


\newglossaryentry{Dockerfile}
{
    name={Dockerfile},
    description={Base file with all installations and configurations for a container image}
}

\newglossaryentry{Docker image}
{
    name={Docker image},
    description={A built system listed under "docker images" which is runnable}
}

\newglossaryentry{Docker daemon}
{
    name={Docker daemon},
    description={Manages and organizes objects inside of the container engine Docker}
}

\newglossaryentry{container engine}
{
    name={container engine},
    description={The foundation for managing and organizing containers}
}

\newglossaryentry{package}
{
    name={package},
    description={Software offered comprimized and with dependencies to other software by different Linux distributions}
}

\newglossaryentry{shell}
{
    name={shell},
    description={Terminal of a Linux/Unix system for entering commands}
}

\newglossaryentry{bash}
{
    name={bash},
    description={Unix shell and command language integrated in all Linux operating systems}
}

\newglossaryentry{issue}
{
    name={issue},
    description={A bug report because of a failure in the software}
}

\newglossaryentry{deployment}
{
    name={deployment},
    description={Fast automated setup of a system with applications and configurations}
}

\newglossaryentry{root permissions}
{
    name={root permissions},
    description={Administrator access permissions with all privileges for a computer system based on Linux/Unix}
}

\newglossaryentry{configuration}
{
    name={configuration},
    description={Saved settings for an application or computer program}
}

\newglossaryentry{configuration management}
{
    name={configuration management},
    description={A process for maintaining computer systems, servers and software in a defined consistent state}
}
\newglossaryentry{mounted}
{
    name={mounted},
    description={Integrated into a system with the mount command}
}

\newglossaryentry{multi-architecture images}
{
    name={multi-architecture images},
    description={Images based on system builds usable for multiple system architectures}
}

\newglossaryentry{cross-compilation}
{
    name={cross-compilation},
    description={Method for compiling code for a different computer system or architecture}
}

\newglossaryentry{JSON}
{
    name={JSON},
    description={The JavaScript Object Notation is an open standard file format for saving data structured}
}

\newglossaryentry{Ubuntu}
{
    name={Ubuntu},
    description={Linux distribution maintained by Canonical}
}

\newglossaryentry{hard disk}
{
    name={hard disk},
    description={The whole operating system with data is written and saved on this component of the computer system}
}

\newglossaryentry{repository}
{
    name={repository},
    description={The location in the internet where software packages are downloadable for installations provided by Linux distributions and other providers}
}

\newglossaryentry{Github}
{
    name={Github},
    description={Software version control system by Microsoft for free usage}
}

\glsaddall
